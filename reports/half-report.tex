\documentclass[a4paper,brazilian,12pt]{article}

\usepackage[brazilian]{babel}
\usepackage[a4paper, left=3cm, right=3cm, top=2.5cm, bottom=2.5cm]{geometry}
\usepackage{graphicx}
\usepackage{minted}
\usepackage{booktabs}
\usepackage{xcolor}

\definecolor{bg}{rgb}{0.95,0.95,0.95}

\title{Elaboração de um software para o desenvolvimento sistemas ópticos dedicado ao ensino da disciplina de Design Óptico no curso de Física}
\author{Vinicius Henrique Pereira Giroto}
\date{\today}

\begin{document}

\maketitle

\section{Introdução}

Este relatório parcial descreve o progresso do projeto de desenvolvimento de um software para o design de sistemas ópticos, focado no ensino da disciplina de Design Óptico no curso de Física. O projeto utiliza a linguagem Rust e a biblioteca wgpu (https://github.com/gfx-rs/wgpu) para implementar um traçador de raios acelerado por GPU.

\section{Desenvolvimento Atual}

\subsection{Traçado de Raios na GPU}

Implementamos um método para traçar raios através de uma lista de superfícies, utilizando a GPU para aceleração. O método calcula os pontos de interseção, direções normais e distâncias.

\subsubsection{Estrutura \texttt{Ray}}

A estrutura \texttt{Ray} representa um raio no espaço 3D.

\begin{minted}{rust}
// Insira aqui a definição da estrutura Ray
\end{minted}

\subsubsection{Estrutura \texttt{Surface}}

A estrutura \texttt{Surface} representa uma superfície óptica.

\begin{minted}{rust}
// Insira aqui a definição da estrutura Surface
\end{minted}

\subsubsection{Estrutura \texttt{Intersection}}

A estrutura \texttt{Intersection} armazena os dados de interseção entre um raio e uma superfície.

\begin{minted}{rust}
// Insira aqui a definição da estrutura Intersection
\end{minted}

\subsection{Cálculo de Parâmetros do Sistema}

Desenvolvemos métodos para calcular parâmetros importantes do sistema óptico, como espessura, curvatura, raio e distância focal. O software é projetado para ser extensível, permitindo a adição de novos parâmetros.

\begin{minted}{rust}
// Insira aqui o código Rust para o cálculo de parâmetros
\end{minted}

\subsection{Função de Mérito}

Atualmente, estamos trabalhando na implementação de métodos para calcular a função de mérito, que será utilizada no processo de otimização do sistema óptico.

\subsubsection{O que é a Função de Mérito?}

A função de mérito (ou função objetivo) é uma função matemática que quantifica o desempenho de um sistema óptico. No contexto da otimização, o objetivo é minimizar (ou maximizar) a função de mérito, ajustando os parâmetros do sistema.

\subsubsection{Para que serve a Função de Mérito?}

A função de mérito serve para guiar o processo de otimização, permitindo que o software encontre a configuração de parâmetros que resulta no melhor desempenho do sistema óptico.

\subsubsection{Descrição do Desenvolvimento}

[Descreva aqui o progresso no desenvolvimento da função de mérito.]

\section{Exemplo de Traçado de Raios}

\begin{center}
    \includegraphics[width=0.8\textwidth]{image.svg}
\end{center}

\begin{center}
    \begin{tabular}{cccccc}
        \toprule
        Surface & Ray & t & Origin & Normal & Direction \\
        \midrule
        1 & 1 & 1.5 & (0, 0, 0) & (0, 0, 1) & (0, 0, 1) \\
        2 & 1 & 2.0 & (0, 0, 1.5) & (0, 0, -1) & (0, 0, 1) \\
        \bottomrule
    \end{tabular}
\end{center}

\section{Próximos Passos}

* Concluir a implementação da função de mérito.
* Desenvolver um algoritmo de otimização para o sistema óptico.
* Criar uma interface gráfica para o software.
* Realizar testes e validação do software.

\section{Conclusão}

O projeto está progredindo conforme planejado. As funcionalidades implementadas até o momento fornecem uma base sólida para o desenvolvimento de um software completo para o design de sistemas ópticos.

\end{document}
